\documentclass[12pt]{article}
\usepackage[utf8]{inputenc}
\usepackage{listings}
\usepackage{xcolor}
\usepackage{graphicx}
\usepackage[a4paper, margin=1in]{geometry}
\usepackage{float}
\usepackage{amsmath}
\usepackage{booktabs}
\usepackage{multirow}
\usepackage{minted}
\linespread{1.5}

\title{Homework 4 Report}
\author{Ezequiel Buck, Esteban Murillo}
\date{\today}

\begin{document}

\maketitle

\section{Part A: Data Preprocessing and Feature Creation}

\subsection{Key Modifications}

\begin{enumerate}
    \item \textbf{Exclude Status X}: All entries with Status = X are ignored because customers with no loans are neither "good" nor "bad" at paying their loans.
    
    \item \textbf{Compute Status Counts}: For each customer, we calculate the total counts of "C", "0", and "1" statuses, as well as total observations (excluding Xs).
    
    \item \textbf{Calculate Fractions}: We compute the fraction of observations that are "C", "0", and "1" for each customer:
    
\begin{minted}[fontsize=\small, linenos]{python}
# Group by ID and calculate fractions for each status
status_fractions = credit_df[credit_df['STATUS'].isin(['C', '0', '1'])].groupby(
    credit_id_col)['STATUS'].value_counts(normalize=True).unstack(fill_value=0)

# Rename columns to indicate they are fractions
status_fractions.columns = [f'{col}_fraction' for col in status_fractions.columns]

# Add these new columns to app_df
app_df = app_df.join(status_fractions, on=app_id_col)
\end{minted}

    \item \textbf{Delinquency Label}: Compute the "Delinquent" column as before "1" if customer has any statuses equal to "2", "3", "4", or "5", and zero otherwise:
    
\begin{minted}[fontsize=\small, linenos]{python}
def is_delinquent(customer_id):
    # Define what counts as a delinquent status
    delinquent_statuses = ['2', '3', '4', '5', 2, 3, 4, 5]
    
    # Get all status records for this customer
    customer_records = credit_df[credit_df[credit_id_col] == customer_id]
    customer_statuses = customer_records['STATUS']
    
    # Check if customer has any delinquent status
    has_delinquent_status = any(
        status in delinquent_statuses for status in customer_statuses)
    
    # Return 1 if delinquent, 0 if not
    return int(has_delinquent_status)

# Apply the function to each record
app_df['Delinquent'] = app_df[app_id_col].apply(is_delinquent)
\end{minted}

\end{enumerate}

\subsection{Terminal Output}

The program produces the following output showing the data processing:

\begin{verbatim}
Value counts of the STATUS column: STATUS
C    442031
0    383120
1     11090
5      1693
2       868
3       320
4       223

Total observations in the credit_df: ID
5005005    61
5022730    61
5061848    61
5061810    61
5061741    61
           ..
5060177     1
5099734     1
5060155     1
5035790     1
5135846     1

Delinquent counts:
 Delinquent
0    32494
1      616
\end{verbatim}

\section{Part B: Feature Exploration}

\subsection{Histograms of Ratio Columns}

We created histograms to show the distribution of the new ratio columns:

\begin{figure}[H]
    \centering
    \includegraphics[width=0.8\textwidth]{../graphs/C_fraction_histogram.png}
    \caption{Histogram of C\_fraction (Closed Account Fractions)}
    \label{fig:c_hist}
\end{figure}

\begin{figure}[H]
    \centering
    \includegraphics[width=0.8\textwidth]{../graphs/0_fraction_histogram.png}
    \caption{Histogram of 0\_fraction (No Payment Due Fractions)}
    \label{fig:0_hist}
\end{figure}

\begin{figure}[H]
    \centering
    \includegraphics[width=0.8\textwidth]{../graphs/1_fraction_histogram.png}
    \caption{Histogram of 1\_fraction (Current Payment Fractions)}
    \label{fig:1_hist}
\end{figure}

\subsection{Bar Plots of Delinquency Rates}

We created bar plots showing delinquency rates for customers in different intervals of these ratio columns:

\begin{figure}[H]
    \centering
    \includegraphics[width=0.8\textwidth]{../graphs/delinquent_vs_C_fraction.png}
    \caption{Delinquency Rate vs. C\_fraction Intervals}
    \label{fig:c_delinquent}
\end{figure}

\begin{figure}[H]
    \centering
    \includegraphics[width=0.8\textwidth]{../graphs/delinquent_vs_0_fraction.png}
    \caption{Delinquency Rate vs. 0\_fraction Intervals}
    \label{fig:0_delinquent}
\end{figure}

\begin{figure}[H]
    \centering
    \includegraphics[width=0.8\textwidth]{../graphs/delinquent_vs_1_fraction.png}
    \caption{Delinquency Rate vs. 1\_fraction Intervals}
    \label{fig:1_delinquent}
\end{figure}

\subsection{Code for Feature Exploration}

\begin{minted}[fontsize=\small, linenos]{python}
# Plot bar chart of the C, 0, 1 ratios of the STATUS column
for col in fractions:
    df.hist(column=[col], bins=50)
    plt.title(col)
    plt.savefig(f'graphs/{col}_histogram.png', dpi=300, bbox_inches='tight')
    plt.show()

# Plot a bar chart of the Delinquent vs. the fractions
for col in fractions:
    plt.figure(figsize=(10, 6))
    df[col] = pd.qcut(df[col], 20, duplicates="drop")
    lm = sns.barplot(data=df, x=col, y="Delinquent")
    plt.xticks(rotation=30, ha='right')
    plt.title(f"Delinquent vs. {col}")
    plt.xlabel(f"{col} Intervals")
    plt.ylabel("Delinquency Rate")
    plt.tight_layout()
    plt.savefig(f'graphs/delinquent_vs_{col}.png', dpi=300, bbox_inches='tight')
    plt.show()
\end{minted}

\section{Part C: Model Training and Evaluation}

\subsection{Logistic Regression Model}

\begin{minted}[fontsize=\small, linenos]{python}
df = pd.get_dummies(df, prefix_sep="_", drop_first=False, dtype=int)
labels = df["Delinquent"]
df = df.drop(columns="Delinquent")

# Shuffle and split into training and test subsets, using random state 2025
train_data, test_data, train_labels, test_labels = \
    sklearn.model_selection.train_test_split(df, labels,
                test_size=0.2, shuffle=True, random_state=2025)

# Standardize the scale of all input columns
train_means = train_data.mean()
train_stds = train_data.std()
train_data = (train_data - train_means) / train_stds
test_data = (test_data - train_means) / train_stds

# Create and train a new logistic regression classifier
model = sklearn.linear_model.LogisticRegression(solver='newton-cg', tol=1e-6)
# Train it with the training data and labels
model.fit(train_data[cols], train_labels)
# Get the prediction probabilities
pred_proba = model.predict_proba(test_data[cols])[:, 1]
\end{minted}

\subsection{Model Performance Evaluation}

\subsubsection{Precision-Recall Curve}

\begin{figure}[H]
    \centering
    \includegraphics[width=0.8\textwidth]{../graphs/precision_recall_curve.png}
    \caption{Precision-Recall Curve}
    \label{fig:pr_curve}
\end{figure}

\subsubsection{ROC Curve}

\begin{figure}[H]
    \centering
    \includegraphics[width=0.8\textwidth]{../graphs/roc_curve.png}
    \caption{ROC Curve}
    \label{fig:roc_curve}
\end{figure}

\subsection{Terminal Output for Model Performance}

\begin{verbatim}
Test AUC score: 0.8556
Train AUC score: 0.8977
\end{verbatim}

\subsection{Comparison with Previous Model}

\textbf{Did the new input columns help improve the predictions?}

The new ratio columns (C\_fraction, 0\_fraction, 1\_fraction) provide additional information about customer payment patterns that were not captured in the original credit\_delinquency.csv. These features help the model better understand:

\begin{itemize}
    \item Payment behavior patterns (proportion of different payment statuses)
    \item Customer credit management history
    \item Risk indicators based on payment consistency
\end{itemize}

The improvement in AUC score and the shape of the ROC curve indicate that these new features contribute to better delinquency prediction.

\section{Part D: Feature Importance Analysis}

\subsection{Feature Coefficient Analysis}

\begin{verbatim}
Top 20 Most Important Features:
                                  Feature  Abs_Coefficient
0                              Unnamed: 0         1.130028
13                             1_fraction         0.381867
7                           DAYS_EMPLOYED         0.307303
51            OCCUPATION_TYPE_Secretaries         0.259890
1                           CODE_GENDER_M         0.225709
42  OCCUPATION_TYPE_High skill tech staff         0.192249
54   OCCUPATION_TYPE_Waiters/barmen staff         0.189574
49          OCCUPATION_TYPE_Realty agents         0.170681
37         OCCUPATION_TYPE_Cleaning staff         0.164098
9                              FLAG_PHONE         0.161765
6                              DAYS_BIRTH         0.157943
45     OCCUPATION_TYPE_Low-skill Laborers         0.145098
43               OCCUPATION_TYPE_IT staff         0.123097
29               NAME_FAMILY_STATUS_Widow         0.122391
22  NAME_EDUCATION_TYPE_Incomplete higher         0.118342
20    NAME_EDUCATION_TYPE_Academic degree         0.117666
41               OCCUPATION_TYPE_HR staff         0.117172
40                OCCUPATION_TYPE_Drivers         0.100317
2                            FLAG_OWN_CAR         0.097493
17         NAME_INCOME_TYPE_State servant         0.089300
\end{verbatim}

\subsection{Analysis of New Column Importance}

\textbf{How important are each of the new columns to your predictor?}

Based on the coefficient magnitudes:

\begin{verbatim}
Importance of Ratio Columns:
       Feature  Abs_Coefficient
13  1_fraction         0.381867
12  0_fraction         0.087761
14  C_fraction         0.018866
\end{verbatim}

\textbf{Does this correlate with the improvement of the results?}

Yes, to a certain point. 1\_fraction greatly improves in the accuracy of the model, being an excellent contribution. This directly correlates to the improvement of the model. However, 0\_fraction and C\_fraction didn't do as good. Even tho they help, these features do not correlate to the improvement of the model as much as we thought they would.
\end{document}
